\documentclass{article}
\usepackage[MeX]{polski}
\usepackage[utf8]{inputenc}
\usepackage{amsfonts}
\usepackage{amsthm}
\usepackage{fullpage}

\title{Rozwiązanie drugiego zadania z~szóstej listy z~Mechaniki Kwantowej}
\author{BP PP RW RP}
\date{\today}

\begin{document}

\maketitle

\begin{enumerate}
\item
Na oscylatorze harmonicznym w~stanie podstawowym
wykonano po kolei dwa pomiary -
najpierw energii kinetycznej, a~następnie energii
całkowitej. Zaniedbując czas pomiędzy pomiarami
podać i~uzasadnić wzór na
prawdopodobieństwo uzyskania
w~wyniku drugiego pomiaru wartości
$ E={7\over 2} \; \bar{}\!\!h \omega$ jeżeli w~wyniku
pierwszego pomiaru uzyskano wynik w~przedziale
$(0,{3\over 2}\;\bar{}\!\!h \omega)$.

\paragraph{Rozwiązanie:}
Energia kinetyczna zależy tylko od pędu, więc mamy
\[
E_k = \frac{p^2}{2m} < \frac{3}{2}\hbar\omega \mathrm{,}
\]
co jest równoważne stwierdzeniu
\[
p \in \left[-\sqrt{3\hbar\omega{}m}, \sqrt{3\hbar\omega{}m}\right] \mathrm{.}
\]
Zatem po pomiarze energii kinetycznej dla cząstki w~stanie $\psi$ jej nowy 
stan będzie równy
\[
|\phi\rangle = \frac{|\phi_0\rangle}{\sqrt{\langle\phi_0|\phi_0\rangle}} \mathrm{,}
\]
gdzie
\[
|\phi_0\rangle = \int\limits_{-\sqrt{3\hbar\omega{}m}}^{\sqrt{3\hbar\omega{}m}}
dp|p\rangle\langle{}p|\psi\rangle \mathrm{.}
\]
Zatem prawdopodobieństwo, że energia całkowita będzie równa 
$\frac{7}{2}\hbar\omega$ wynosi $\langle\psi_3|\phi\rangle$,
gdzie $\psi_3$ oznacza funkcję własną operatora energii odpowiadającą wartości
własnej $\frac{7}{2}\hbar\omega$.

\item Mierząc położenie cząstki kwantowej przy pomocy detektora idealnego 
umieszczonego na odcinku
$[a,b]$, otrzymano wynik negatywny. Podać wzór na funkcję falową
cząstki (w~reprezentacji położeniowej) tuż po pomiarze, jeżeli tuż
przed pomiarem jej stan opisany był unormowaną funkcją falową
$\psi(x)$.

\paragraph{Rozwiązanie:}
Jeżeli detektor idealny nie zarejestrował cząstki na przedziale $[a,b]$, to
funkcja falowa po obserwacji na tym przedziale powinna znikać. Jest to sytuacja 
równoważna sytuacji w~której cząstkę zarejestrował drugi detektor idealny 
badający istnienie cząstki na dopełnieniu przedziału $[a,b]$. Zatem, po 
obserwacji układ będzie w~stanie:
\[
|\phi\rangle = \frac{|\phi_0\rangle}{\sqrt{\langle\phi_0|\phi_0\rangle}} 
\mathrm{,} \quad \mathrm{gdzie} \quad
|\phi_0\rangle = \int\limits_{-\infty}^a dx|x\rangle\langle{}x|\phi\rangle + 
\int\limits_b^\infty dx|x\rangle\langle{}x|\phi\rangle \mathrm{.}
\]
Wiedząc, że funkcje własne operatora położenia wyrażają się przez deltę Diraca,
otrzymujemy:
\[
\phi_0(x) = \int\limits_{-\infty}^a\delta(x-y)\psi(y)dy + 
\int\limits_b^\infty\delta(x-y)\psi(y)dy = 
\left\{\begin{array}{l l}
	0 & x\in[a,b]\\
	\psi(x) & x\notin [a,b]
\end{array}\right.  \mathrm{.}
\]
Widać, że zgodnie z~naszymi intuicjami funkcja ta znika na przedziale $[a,b]$.

\item
Rozważmy dwa pomiary położenia cząstki, jeden na odcinku $[a_1,a_2]$, drugi na odcinku
$[b_1,b_2]$. Zakładając pozytywny wynik obu pomiarów, zbadać czy kolejność ich wykonywania
ma wpływ na stan końcowy układu. 

\paragraph{Rozwiązanie:}
Jak już zauważyliśmy, tuż po pozytywnym pomiarze położenia cząstki będącej w~stanie $\psi$
na przedziale $[a,b]$, nowy stan ma postać:
\[
\phi(x) = \left\{\begin{array}{l l}
	A\psi(x) & x \in [a,b] \\
	0 & x \notin [a,b]
\end{array}\right. \mathrm{,}
\]
gdzie $A$ to pewna rzeczywista dodatnia stała normująca.

Zatem zaczynając od stanu $\psi$,po wykonaniu pierwszego pomiaru 
na przedziale $[a_1,a_2]$ mamy stan
\[
\phi_1 = \left\{\begin{array}{l l}
	A_1\psi(x) & x \in [a_1,a_2]\\
	0 & x \notin [a_1,a_2]
\end{array}\right. \mathrm{,}
\]
a~następnie po wykonaniu drugiego pomiaru na przedziale $[b_1,b_2]$ otrzymujemy stan
\[
\phi_2 = \left\{\begin{array}{l l}
	B_1\phi_1(x) & x \in [b_1,b_2]\\
	0 & x \notin [b_1,b_2]
\end{array}\right. =
\left\{\begin{array}{l l}
	A_1B_1\psi(x) & x \in [a_1,a_2] \cap [b_1,b_2]\\
	0 & x \notin [a_1,a_2] \cap [b_1,b_2]
\end{array}\right. \mathrm{.}
\]
Wykonując te same pomiary, ale w~odwrotnej kolejności otrzymamy kolejno stany
\[
\phi_1' = \left\{\begin{array}{l l}
	B_2\psi(x) & x \in [b_1,b_2]\\
	0 & x \notin [b_1,b_2]
\end{array}\right.
\]
\[
\phi_2' = \left\{\begin{array}{l l}
	A_2\phi_1'(x) & x \in [a_1,a_2]\\
	0 & x \notin [a_1,a_2]
\end{array}\right. =
\left\{\begin{array}{l l}
	A_2B_2\psi(x) & x \in [a_1,a_2] \cap [b_1,b_2]\\
	0 & x \notin [a_1,a_2] \cap [b_1,b_2]
\end{array}\right. \mathrm{.}
\]

Widać, że stałe $A_1B_1$ i $A_2B_2$ są rzeczywiste dodatnie, oraz normują
tą samą funkcję, więc muszą być równe. Zatem stan końcowy nie zależy od kolejności
pomiarów położenia.

\end{enumerate}

\end{document}
