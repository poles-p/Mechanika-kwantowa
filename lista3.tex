\documentclass{article}
\usepackage[MeX]{polski}
\usepackage[utf8]{inputenc}
\usepackage{amsfonts}
\usepackage{amsthm}
\usepackage{fullpage}

\title{Rozwiązanie drugiego zadania z~trzeciej listy z~Mechaniki Kwantowej}
\author{Piotr Polesiuk, Bartłomiej Pytko}
\date{\today}

\begin{document}

\maketitle

Niech $A\colon\mathbb{C}^n\to \mathbb{C}^n$
będzie operatorem hermitowskim. Udowodnić następujące twierdzenia
\begin{enumerate}
\item
Jeżeli podprzestrzeń $W\subseteq \mathbb{C}^n$ jest podprzestrzenią
niezmienniczą operatora $A$ to jej ortogonalne dopełnienie
$W^\perp$ jest również podprzestrzenią niezmienniczą $A$.

\begin{proof}
Operator $A$ jest hermitowski, tzn. dla dowolnych wektorów 
$x,y\in\mathbb{C}^n$ zachodzi 
\mbox{$\langle{}Ax|y\rangle=\langle{}x|Ay\rangle$}.

Weźmy dowolny wektor $u\in W^\perp$. Z~definicji ortogonalnego dopełnienia 
spełnia on $\forall w\in W.\langle{}u|w\rangle=0$. Pokażę, że własność ta 
zachodzi również dla $Au$. W~tym celu weźmy dowolne $w\in W$. Wtedy
\[
\langle{}Au|w\rangle = \langle{}u|Aw\rangle = 0
\]
Pierwsza równość wynika z~hermitowskości operatora $A$, zaś równość 
druga wynika z~niezmienniczości podprzestrzeni $W$ ($Aw \in W$) oraz 
z~definicji ortogonalnego dopełnienia.
\end{proof}

\item
Istnieje baza ortonormalna złożona z wektorów własnych operatora $A$.

\item
Przestrzeń $\mathbb{C}^n$ jest ortogonalną sumą prostą podprzestrzeni własnych
operatora $A$.
\end{enumerate}

\end{document}
