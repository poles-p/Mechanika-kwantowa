\documentclass{article}
\usepackage[MeX]{polski}
\usepackage[utf8]{inputenc}
\usepackage{amsfonts}
\usepackage{amsthm}
\usepackage{fullpage}

\title{Rozwiązanie drugiego zadania z~trzeciej listy z~Mechaniki Kwantowej}
\author{BP PP RW RP}
\date{\today}

\begin{document}

\maketitle

Niech $A\colon\mathbb{C}^n\to \mathbb{C}^n$
będzie operatorem hermitowskim. Udowodnić następujące twierdzenia
\begin{enumerate}
\item
Jeżeli podprzestrzeń $W\subseteq \mathbb{C}^n$ jest podprzestrzenią
niezmienniczą operatora $A$ to jej ortogonalne dopełnienie
$W^\perp$ jest również podprzestrzenią niezmienniczą $A$.

\begin{proof}
Operator $A$ jest hermitowski, tzn. dla dowolnych wektorów 
$x,y\in\mathbb{C}^n$ zachodzi 
\mbox{$\langle{}Ax|y\rangle=\langle{}x|Ay\rangle$}.

Weźmy dowolny wektor $u\in W^\perp$. Z~definicji ortogonalnego dopełnienia 
spełnia on $\forall w\in W.\langle{}u|w\rangle=0$. Pokażę, że własność ta 
zachodzi również dla $Au$. W~tym celu weźmy dowolne $w\in W$. Wtedy
\[
\langle{}Au|w\rangle = \langle{}u|Aw\rangle = 0
\]
Pierwsza równość wynika z~hermitowskości operatora $A$, zaś równość 
druga wynika z~niezmienniczości podprzestrzeni $W$ ($Aw \in W$) oraz 
z~definicji ortogonalnego dopełnienia.
\end{proof}

\item
Istnieje baza ortonormalna złożona z wektorów własnych operatora $A$.

\begin{proof}
Przez indukcję po $n$.
\begin{itemize}
\item Dla $n=0$ zbiór pusty w~sposób trywialny jest bazą ortonormalną.
\item Załóżmy, że teza twierdzenia zachodzi dla każdego operatora 
hermitowskiego działającego nad~$\mathbb{C}^n$. 
Weźmy operator $A\colon\mathbb{C}^{n+1}\to\mathbb{C}^{n+1}$.
Jego wielomian charakterystyczny jest wielomianem stopnia $n+1$ nad
ciałem liczb zespolonych, więc z~zasadniczego twierdzenia algebry
operator $A$ ma conajmniej jedną wartość własną $\lambda$ oraz wektor 
własny $w$. Oczywiście wektor ten można wybrać tak by $||w||=1$.

Przestrzeń $\mathbb{C}^{n+1}$ można przedstawić jako
\[
\mathbb{C}^{n+1} = \mathbb{C}w \oplus \left(\mathbb{C}w\right)^\perp
\]
(gdzie zapis $\mathbb{C}w$ oznacza 
$\left\{\alpha{}w | \alpha\in\mathbb{C}\right\}$), a~każdy wektor $v$
przedstwić w~sposób jednoznaczny jako sumę $v=\alpha{}w+v'$, gdzie 
$v'\in \left(\mathbb{C}w\right)^\perp$. Zauważmy jeszcze, że podprzestrzeń
$\mathbb{C}w$ jest niezmiennicza (bo $w$ jest wektorem własnym), a~na
mocy poprzednio udowodnionego twierdzenia, podprzestrzeń 
$\left(\mathbb{C}w\right)^\perp$ również jest niezmiennicza.

Rozważmy działanie operatora $A$ na dowolnym wektorze $v$:
\[
Av = A\left(\alpha{}w+v'\right) = A\left(\alpha{}w\right) + Av' = 
\alpha\lambda{}w + A'v'
\]
gdzie operator $A'$ to operator $A$ obcięty do podprzestrzeni
$\left(\mathbb{C}w\right)^\perp$, wartości $A'$ również są z~tego zbioru
(z~niezmienniczości dziedziny).

Ale $\dim\left(\mathbb{C}w\right)^\perp = n$, czyli przestrzeń 
$\left(\mathbb{C}w\right)^\perp$ jest izomorficzna z~przestrzenią $\mathbb{C}^n$,
więc na mocy założenia indukcyjnego ma bazę ortonormalną $E$, złożoną z~wektorów
własnych operatora $A'$, które również są wektorami własnymi operatora $A$.

Wektor $w$ jest prostopadły do wszystkich wektorów 
z~$\left(\mathbb{C}w\right)^\perp$, w~szczególności do wektorów z~$E$, więc układ
$E\cup\left\{w\right\}$ tworzy bazę ortonormalną przestrzeni $\mathbb{C}^{n+1}$
złożoną z~wektorów własnych operatora $A$.
\end{itemize}
\end{proof}

\item
Przestrzeń $\mathbb{C}^n$ jest ortogonalną sumą prostą podprzestrzeni własnych
operatora $A$.
\end{enumerate}

\begin{proof}Natychmiast z~poprzedniego twierdzenia. 
Niech $\left\{v_{\lambda i} | \lambda\in{\rm spec}A | i=1,\ldots,m_\lambda\right\}$,
gdzie $m_\lambda$ oznacza krotność geometryczną wartości własnej $\lambda$,
będzie wyżej wprowadzoną bazą ortonormalną. Zatem
\[
\mathbb{C}^n = \bigoplus\limits_{\lambda\in{\rm spec}A}\bigoplus\limits_{i=1}^{m_\lambda}
	\mathbb{C}v_{\lambda i} =
\bigoplus\limits_{\lambda\in{\rm spec}A}V_\lambda
\]
Ta ostatnia równość wynika z~faktu, że wektory
$\left\{v_{\lambda i}\right\}_{i=1}^{m_\lambda}$ rozpinają podprzestrzeń własną $V_\lambda$
odpowiadającą wartości własnej $\lambda$. 

Ortogonalność tej sumy zapewniona jest przez ortonormalność bazy.
\end{proof}

\end{document}
