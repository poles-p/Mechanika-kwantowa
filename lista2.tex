\documentclass{article}
\usepackage[MeX]{polski}
\usepackage[utf8]{inputenc}
\usepackage{amsfonts}
\usepackage{fullpage}

\title{Rozwiązanie drugiego zadania z drugiej listy z~Mechaniki Kwantowej}
\author{Piotr Polesiuk, Bartłomiej Pytko}
\date{\today}

\newcommand*{\pd}[2]{\frac{\partial #1}{\partial #2}}
\newcommand*{\pdd}[3]{\frac{\partial^2 #1}{\partial #2 \partial #3}}

\begin{document}

\maketitle

\paragraph{Zadanie 2, podpunkt (a).}

Nawiasy Poissona definiujemy jako
\[
\{F,G\} \equiv \sum_{i=0}^s
\left(\pd{F}{p_i}\pd{G}{q_i} - \pd{F}{q_i}\pd{G}{p_i}\right)
\]
gdzie $s$ to liczba stopni swobody, a~$p_i$ oraz $q_i$ to pędy 
i~współrzędne uogólnione.

Jako, że przy rozwijaniu definicji zagnieżgdżonych nawiasów Poissona mogą
pojawiać się bardzo długie wyrażenia ($\{F,\{G,H\}\}$ po rozwinięciu
ma aż 8 składników postaci $\pd{A}{x}\pd{B}{y}$), wprowadzimy następującą
notację:
\[
x_i^0 \equiv p_i \qquad x_i^1 \equiv p_i
\]
Wtedy nawiasy Poissona wyrażają się w~następujący sposób:
\[
\{F,G\} = \sum_{i=0}^s\sum_{a=0}^1 (-1)^a \pd{F}{x_i^a}\pd{G}{x_i^{1-a}}
\]
a~zatem
\[
\{F,\{G,H\}\} = \sum_{i=0}^s\sum_{a=0}^1 (-1)^a \pd{F}{x_i^a}\pd{}{x_i^{1-b}}
\sum_{j=0}^s\sum_{b=0}^1 (-1)^b \pd{G}{x_j^b}\pd{H}{x_j^{b-1}}
\]
\[
= \sum_{i,j=0}^s\sum_{a,b=0}^1 (-1)^{a+b}
\pd{F}{x_i^a} \pdd{G}{x_i^{1-a}}{x_j^b} \pd{H}{x_j^{1-b}}
+ \sum_{i,j=0}^s\sum_{a,b=0}^1 (-1)^{a+b}
\pd{F}{x_i^a} \pd{G}{x_j^b} \pdd{H}{x_i^{1-a}}{x_j^{1-b}}
\]
Po przeindeksowaniu w~drugiej sumie i~zamianie kilku czynników miejscami mamy
\begin{equation}\label{poissonXX}
\{F, \{G,H\}\}
= \sum_{i,j=0}^s\sum_{a,b=0}^1 (-1)^{a+b}
\pd{F}{x_i^a} \pdd{G}{x_i^{1-a}}{x_j^b} \pd{H}{x_j^{1-b}}
- \sum_{i,j=0}^s\sum_{a,b=0}^1 (-1)^{a+b}
\pd{F}{x_i^a} \pdd{H}{x_i^{1-a}}{x_j^{b}} \pd{G}{x_j^{1-b}}
\end{equation}
Połóżmy
\[
X_{FGH} \equiv \sum_{i,j=0}^s\sum_{a,b=0}^1 (-1)^{a+b}
\pd{F}{x_i^a} \pdd{G}{x_i^{1-a}}{x_j^b} \pd{H}{x_j^{1-b}}
\]
Taki obiekt ma następujące własności
\begin{eqnarray*}
(i)  && X_{FGH} = X_{HGF} \\
(ii) && \{F,\{G,H\}\} = X_{FGH} - X_{GHF} \\
\end{eqnarray*}
Własność $(i)$ otrzymujemy przez prostą zamianę indeksów,
a~własność $(ii)$ natychmiast wynika z~rówania~(\ref{poissonXX}) oraz
własności $(i)$.

Teraz łatwo można udowodnić własność nawiasów Poissona z~zadania:
\[
\{F,\{G,H\}\} + \{H,\{F,G\}\} + \{G,\{H,F\}\} =
X_{FGH} - X_{GHF} + X_{HFG} - X_{FGH} + X_{GHF} - X_{HFG} = 0
\]

\end{document}
