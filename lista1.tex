\documentclass{article}
\usepackage[MeX]{polski}
\usepackage[utf8]{inputenc}
\usepackage{amsfonts}
\usepackage{fullpage}

\newcommand*{\intd}{\mathrm{\textbf{d}}}

\title{Rozwiązania kilku zadań z~pierwszej listy zadań z~Mechaniki Kwantowej}
\author{Piotr Polesiuk}
\date{\today}

\begin{document}

\maketitle

\paragraph{Zadanie 2.}

Całkowite natężenie promieniowania (po wszystkich częstościach) można wyraźić 
całką:
\[
\psi(T) = \int\limits_0^\infty \psi(\nu, T) \intd\nu
\]
Wstawiając wyrażenie na $\psi(\nu, T)$ z~prawa Plancka:
\[
\psi(T) = \int\limits_0^\infty \frac{8\pi\nu^2}{c^2} 
	\frac{h\nu}{e^\frac{h\nu}{kT} - 1} \intd\nu
\]
oraz całkując przez podstawienie:
\[
x = \frac{h\nu}{kT}
\quad
\nu = \frac{kT}{h}x
\quad
\intd\nu = \frac{kT}{h}\intd x
\]
otrzymujemy
\[
\psi(T) 
= \int\limits_0^\infty \frac{8\pi k^2 T^2}{c^2 h^2}x^2
	\frac{kT}{h}\frac{hx}{e^x - 1}\frac{kT}{h}\intd x
= T^4\frac{8\pi k^4}{c^2 h^3}
	\int\limits_0^\infty x^3\frac{1}{e^x - 1}\intd x
\]
Ta ostania całka przyjmuje stałą wartość, więc nie zależy od temperatury.
Otrzymaliśmy więc prawo Stefana--Boltzmanna:
\[
\psi(T) = \sigma T^4
\]
gdzie
\[
\sigma = \frac{8\pi k^4}{c^2 h^3}
	\int\limits_0^\infty x^3\frac{1}{e^x - 1}\intd x
\]

\end{document}
